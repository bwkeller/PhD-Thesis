On April 26, 1920, Harlow Shapley and Heber Curtis took the stage at the
Smithsonian National Museum of History to debate each other on a topic of of
seemingly only academic interest: the nature of so-called ``Spiral Nebulae''.
The two eminent astronomers argued whether these nebulae were contained within
the Milky Way.  The implication if they lived within the Galaxy was that the
Galaxy was The Universe.  If they were not, the universe was far vaster than
most imagined.  This academic debate, like the Papal disagreement with Galileo
three hundred years earlier, was poised to reshape our understanding of the
vastness of the cosmos, and our insignificant place within it.

Today, nearly a century later, we no longer refer to these objects as ``Spiral
Nebulae'', but instead give them the name first reserved only for our own:
Galaxies.  Our Milky Way is but one of trillions of galaxies in the universe,
filling a volume that stretches out thousands of Megaparsecs in space and
billions of years in time.  These galaxies contain nearly all of the stars and
planets in the universe, much of the universe's gas, and are the places where
the former are constructed from the latter.

\section{The Cosmological Context of Galaxy Formation}
The modern picture of galaxy formation sits within the context of a $\Lambda$
Cold Dark Matter  ($\Lambda CDM$) cosmology.  This is a universe which has a
no curvature, the majority of its mass-energy in a cosmological constant
$\Lambda$, and most of its matter in a collisionless form (Dark matter).  
Roughly $5\%$ of the universe exists as baryons, which began as a mostly
isothermal gas of hydrogen and helium.  

The formation of galaxies within the universe began with the gravitational
collapse of small, linear density perturbations that were likely seeded by
quantum fluctuations amplified by inflation.  These perturbations grow through
gravitational collapse.  Until $z\sim1100$, only dark matter perturbations were
able to grow.  Prior to this, baryons were coupled to photons, causing density
perturbations to simply oscillate as stable sound waves.  This early collapse of
matter meant that when decoupling occured at $z\sim1100$.
\section{Numerical Simulations of Galaxies}
\section{The Importance of Feedback}
\begin{figure}
    \includegraphics[width=\textwidth]{M82.eps}
    \caption{Massive outflows in M82}{The evidence for feedback is no clearer
    than here in the Cigar Galaxy, M82.  The Hubble Space Telescope reveals
    massive outflows of hot ionized gas through the red $H\alpha$ emission seen
    as gas is blasted out of the galaxy in a bipolar flow.}
\end{figure}
