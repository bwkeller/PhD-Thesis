In this thesis, we have explored how supernovae, when modeled with a robust,
physically-motivated method, can impact the growth and evolution of galaxies.
We introduced a novel model for SN feedback that takes into account the
important effects of previously ignored physics: the evaporation of cold gas
driven by thermal conduction.  This model allowed us to see how SN can (and
cannot!) regulate the flow of gas into and out of galaxies, where it ultimately
forms stars.  We have also shown that the current $\Lambda CDM$ cosmological
model is consistent with recent observations of galaxy kinematics.

In Chapter 2, we presented a detailed description of the new superbubble model,
and implemented it in the SPH code {\sc Gasoline2}.  We rely on the strongly
self-limiting process of thermal evaporation \citep{Cowie1977}, where cold mass
is evaporated into a hot bubble with a mass flux given by $ \frac{{\rm d
}M_b}{{\rm d}t} = \frac{4\,\pi\mu}{25k_B}\kappa_0\,\frac{\Delta T^{5/2}}{\Delta
x}\ A$. This allows us to develop a new model for SN feedback, that gives
realistic estimates for cooling in unresolved hot bubbles, and accurately
calculates the mass loading of SN energy.  We demonstrated that this model is
insensitive to resolution, unresolved ISM structure, and uncertain (turbulent)
magnetic fields.  Finally, after investigating the behaviour of individual
bubbles, we showed that superbubble feedback was more effective at both
regulating star formation and at driving galactic winds.

Chapter 3 then applied this new model to the cosmological simulation of a Milky
Way like galaxy.  We compared the same galaxy, simulated four times with
different feedback models (no feedback, the older \citet{Stinson2006} blastwave model,
and superbubble feedback with $10^{51}\;\rm{erg}/SN$ and
$2\times10^{51}\;\rm{erg}/SN$).  The results of this comparison found that even
though the galaxy massively overproduced stars (a result previously seen in
\citealt{Stinson2010}) when simulated with blastwave feedback (or no feedback),
the same SN energy, when simulated using our new superbubble model, was able to
produce a realistic stellar mass fraction for the galaxy's entire history.  Not
only did the galaxy simulated with superbubble feedback have the correct
stellar mass, we showed that it also was effectively bulgeless, with a
completely flat rotation curve.  These two results were ultimately caused by the
same process.  At high redshift, superbubble feedback efficiently drives highly
mass loaded winds, expelling gas from the ISM of the galaxy.  As the gas that
forms a stellar bulge was primarily accreted at high redshift, this meant that
these winds preferentially removed low angular momentum material, while
preserving the gas with high angular momentum.  Thus, superbubble
feedback was able to naturally prevent both runaway star formation and the
formation of a bulge through these winds.

Chapter 4 took these results to the limit to examine whether SN alone are able
to regulate the formation of stars in galaxies both heavier and lighter than the
Milky Way.  By simulating and examining a new sample of 18 galaxies (the
McMaster Unbiased Galaxy Simulations 2), we were able to demonstrate a critical
mass at which SN feedback breaks down as a regulator of gas accretion and star
formation.  Galaxies that exceed this mass, $(\sim 10^{12}\;\Msun)$, will produce
more stars than real galaxies, and produce much of these stars in a dense
stellar bulge.  All of this arises due to a fundamental relation between the
mass loading of superbubble driven winds and the mass of the galaxy (either its
halo or its disc).  When the escape velocity of the disc exceeds the sound speed
of superbubble-heated gas (which has a typical mass loading of $\eta=10$), only
outflows which have entrained less material are able to escape the galaxy,
leading to decreasing efficiency in ejecting material, and a shutdown of
SN regulation.  The fact that this shutdown occurs exactly where powerful AGN
are observed, and that it results in a vigorous transport of gas to the nucleus
of the galaxy make this result strong evidence that this shutdown acts as a kind
of ``hand-off'' between SN and AGN feedback.

Chapter 5 applied the MUGS2 sample of galaxies we produced in
\citet{Keller2016a} and used them to examine the radial acceleration relation
(RAR) that was derived by \citet{McGaugh2016} from observations of the SPARC
\citep{Lelli2016} catalog.  Contrary to the assertions of some
\citep{Milgrom2016}, the tight relation seen in the SPARC data does not require
new fundamental physics (such as self-interacting dark matter, modified gravity,
or MOND).  Instead, the simple combination of dissipational collapse, together
with angular momentum conservation produces the RAR. This means that even
galaxies where SN fails to regulate their star formation (or even where SN are
omitted altogether) fall on the observed relation.  This was the first published
demonstration of this in simulation, using untuned simulations made prior to the
publication of \citet{McGaugh2016}.

This thesis has shown that, with correct modelling, SN feedback can regulate the
growth of galaxies up to the mass of the Milky Way, but no further.  In galaxies
up to this mass, SN efficiently drive galactic winds, primarily at
high redshift, that act to limit the availability of low angular momentum gas,
slowing the formation of stars and the stellar bulge. Without the
feedback from SMBHs, galaxies more massive than ours form far too many stars,
and primarily form those stars in a concentrated central bulge.  This transition
from SN to AGN feedback occurs due to a rapid drop in the effectiveness of
SN-driven outflows.  Finally, we showed that independent of the detailed
internal processes of galaxy evolution, the acceleration relation observed by
\citet{McGaugh2016} is simply a natural consequence of galaxy formation.

\section{Results in Context}
Developing a versatile, accurate, resolution insensitve model for feedback has
been a challenge in galaxy simulations for decades.  While a number of recent
models have been developed that are able to produce galaxies which match some
observables, all of these models have relied on purely numerical features to
enhance the effectiveness of feedback and counter the effects of limited
resolution.  The \citet{Keller2014} presented in Chapter 2 is the first model of
its kind that uses a first-principles approach to prevent overcooling.  Beyond
merely ``sufficient'' feedback, the superbubble model also incorporates physics
that previously have been omitted.  Thermal evaporation is a key process for
mixing cold ISM with hot feedback ejecta.  Prior to \citet{Keller2014}, this
process was completely ignored, and the local mass-loading of feedback was set
purely as a numerical parameter.  

As we showed in \citet{Keller2015}, including thermal evaporation in simulations
of a Milky Way like galaxy will drastically change the effectiveness of
SN-driven galactic winds.  Removing gas from the galaxy is the only effective
mechanism for regulating star formation on cosmic timescales.  To prevent the
formation of bulges larger than the ones seen in the local universe, bulge
forming gas (low-angular momentum gas) must be removed from the ISM; because
SN-driven winds are most effective at high redshift, when this material is
mostly accreted, the outflows driven from the galaxy preferentially remove
bulge-forming gas.  An important conclusion of this study is that much of the
past work showing that SN feedback was unable regulate star formation without
the addition of ``early feedback'' from radiation pressure, stellar winds, or
photoionization was likely underestimating the effectiveness of SN.  Since many
of these studies used simpler, more numerically-sensitive feedback models that
also omitted thermal evaporation, they were unable to effectively drive the
correct amount of mass into hot, high entropy gas that could leave the galaxy.
This study tells us that the disc-dominant morphology and low stellar fraction
in galaxies like our own all arise from the same mechanism:  heavily mass
loaded, high redshift winds.

\section{Future Directions}
These results will be valuable for many new research projects.  The superbubble
model is a robust tool for simulating galaxies, and will become a key part of
future studies using the simulation codes {\sc Gasoline2} and {\sc ChaNGa}.  The
MUGS2 sample is a useful new set of simulations for answering questions
involving the evolution of Milky Way-like galaxies (as evidenced by our use of
it in Chapter 5).  These two resources will be valuable to both myself and other
researchers for years to come. The four publications presented here point
towards a number of questions I plan to answer during my postdoctoral research.  

While observational evidence abounds for the existence of SMBHs
\citep{Kormendy2013}, and the impact of the AGNs they power
\citep{Veilleux2005}, there is still great uncertainty as to the details of how
they regulate their own growth, and the growth of the galaxies they live in.
\citet{Keller2016a} showed evidence that a transition occurs between galaxies
regulated by SN feedback to AGN feedback as they grow heavier than the Milky
Way.  How this handoff occurs remains widely unexplored.  Using my superbubble
feedback, along with new models for SMBH growth \citep{Hopkins2010} and
migration \citep{Tremmel2015}, we will for the first time be able to simulate
the growth of SMBHs in galaxies with first-principles, physically motivated
physics.  This will allow us to study the full gamut of galaxies that exist, and
study how the tight scaling relations between stellar bulges and SMBHs might
arise.

The CGM gas that surrounds galaxies is the interface between the star forming
disc and the galaxy's hot, gaseous halo (and the cosmic web beyond).  Any
material that accretes from the halo must first pass through the CGM, and any
material that leaves the ISM must do the same.  Despite the importance
of this component of the galaxy, observations of gas in the CGM are difficult.
This gas tends to be too cool \& diffuse to be a strong emitter of X-rays, and
absorption line studies require the lucky coincidence of a background quasar
\citep{Weiner2009}.  In addition to this, we know that the CGM contains complex,
multiphase structure, including cool neutral clouds \citep{Wakker1997},
molecular gas \citep{Stark1984}, and dust .  How this cold material can persist in a
CGM that is much hotter and that may be shearing past at supersonic velocities
is currently an open question.  Analytic calculations \& high resolution
simulations suggest that these clouds should be disrupted and mixed on
timescales far shorter than is required to ballistically launch them into the
CGM.  In light of this, why do we see these clouds?  They may be formed in-situ,
or may be launched by the fragmentation of superbubble shells, already
accelerated to high velocities during the momentum-conserving snowplow phase of
the bubble's growth.  High resolution simulations of superbubble breakout from
the ISM will be able to tell us if this is indeed the origin of these clouds.
Understanding the details of the multiphase CGM is key to
understanding how galaxies accrete and expel gas, as thermal instability in the
CGM (which depends heavily on its density, temperature, and metallicity) may be
the mechanism by which spiral galaxies today aquire the gas required to
form new stars \citep{Marasco2012}.  This work should allow us to understand the
origin of the relation between disc mass and outflow mass loadings seen in
\citet{Keller2016a}.


Large volume simulations like Illustris \citep{Vogelsberger2014b} and EAGLE
\citep{Schaye2015} can show how whole populations of galaxies evolved from the
earliest linear structures.  While these simulations contain models for star
formation, feedback from both stars and AGN, as well as radiative cooling, these
have all been fine-tuned to fit observed scaling relations.  This limits the
ability of these simulations to explain the origins of the observations they
have been tuned to fit.  My superbubble model contains no free/tuned parameters,
and accurately captures the evolution of feedback-heated gas as it evaporates
cool material that surrounds it.  It is the only feedback model currently in
existence that captures the important physical process of thermal conduction and
evaporation.  It will allow us to build a statistically significant sample of
physically realistic galaxies.  With first-principles, physically motivated
models for the unresolved physics in our simulations, we will be able to study
the actual physics involved in shaping both individual galaxies and the
populations that we observe. 
\bibliographystyle{mnras}
\bibliography{library}
